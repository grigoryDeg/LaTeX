\documentclass{article}
\usepackage[utf8]{inputenc}
\usepackage[russian]{babel}
\usepackage{amsmath}
\usepackage{hyperref} 
\numberwithin{equation}{section}
\newenvironment{Proof}
	{\par\noindent{ДОКАЗАТЕЛЬСТВО.}}
	{\hfill$\scriptstyle$}

\begin{document}
\title{Параграф, оформленный в  \LaTeX{}}
\author{Григорий Дегтярев}
\maketitle
\tableofcontents
\newpage
\section{4.1 Действие группы на множестве}
Будем говорить, что группа $G$ действует на множестве $D$, если каждому элементу $g$ группы $G$ поставлено в соответствие взаимоодннозначное отображение $\varphi(g)$ множества D в себя так, что $\varphi(g_1g_2)=\varphi(g_1)\varphi(g_2)$ для любых $g_1$  и $g_2$ из $G$. Иначе говоря группа $G$ действует на множестве $D$, если определен гомоморфизм $\varphi$ группы $G$ в множество взаимооднозначных отображений множества $D$ в себя. Рассматривая далее действие группы $G$ на множестве $D$, будем опускать символ гомоморфизма и будем рассматривать элементы группы $G$ непосредственно как преобразования множества $D$: результат действия элемента $g$ группы $G$ на элементе $d$ множества $D$ будем обозначать через $g(d)$ или $gd$. \\
Пусть группа $G$ действует на множестве $D$. Стабилизатором элемента $d_0$ из $D$  называется множество $St(d_0) =   \{g \in  G|g(d_0) =  d_0 \}$ . Орбитой элемента $d_0$  из $D$  называется  множество  $Or(d_0) = \{d\in  D|d=g(d_0)$, где $g \in  G \}$, число элементов орбиты называется ее длиной. Известно, что стабилизатор любого элемента $d$ является подгруппой в группе $G$. \\ 
\newtheorem{Def}{Лемма 4.}
		\begin{Def}\label{fourone}
			Пусть конечная группа $G$ действует на конечном множестве $D$. Тогда для любого $d$ из $D$
		\end{Def}
					$$|Or(d)|\cdot|St(d)| = |G|.$$
\begin{Proof}
Покажем, что длина орбиты произвольного элемента $d$ из $D$ равна числу смежных классов группы $G$ по подгруппе $St(d)$. Для этого достаточно показать, что элементы из одного смежного класса переводят $d$ в один и тот же элемент множества $D$, а элементы из разных смежных классов --- в разные элементы множества $D$. \\
Если $g_1$ и $g_2$ лежат в одном и том же смежном классе группы $G$ по подгруппе $St(d)$, то $g_2$ = $g_1s$, где $s \in St(d)$. Поэтому
$$g_2(d) = g_1s(d) = g_1(s(d)) = g_1(d),$$
т. е. элементы из одного и того же смежного класса группы $G$ по подгруппе $St(d)$ отображают $d$ в один и тот же элемент множества $D$. Теперь покажем, что элементы из разных смежных классов группы $G$ по подгруппе $St(d)$ отображают $d$ в разные элементы множества $D$. Допустим, что $g_1(d) = g_2(d)$, тогда 
$$d = g_1^{-1}g_1(d) = g_1^{-1}(g_1(d)) = g_1^{-1}(g_2(2)) = g_1^{-1}g_2(d),$$
и, следовательно, $g_1^{-1}g_2 \in St(d)$. Но тогда в $St(d)$ найдется такой элемент $s$, что $g_2 = g_1s$, и поэтому $g_1$ и $g_2$ лежат в одном и том же смежном классе группы $G$ по подгруппе $St(d)$. \\
Так как число смежных классов группы $G$ по подгруппе $St(d)$ равно
$|G|/|St(d)|$, а длина орбиты элемента $d$ равна числу смежных классов группы $G$ по подгруппе $St(d)$, то $|Or(d)|\cdot|St(d)| = |G|$. Лемма доказана.
\end{Proof} 
 
\end{document}
